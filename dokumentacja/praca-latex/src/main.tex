\documentclass{article}
\usepackage{polski}
\usepackage[utf8]{inputenc}
\usepackage[margin=1in,left=1.5in,includefoot]{geometry}
\usepackage{fancyhdr}
\pagestyle{fancy}

\begin{document}

\begin{titlepage}

\title{Praca inżynierska}
\author{Kamil Susek}
\date{September 2020}

\maketitle
\end{titlepage}
\section{Wprowadzenie}
\newpage
\section{Blockchain}
\newpage
\section{Analiza dziedziny}
Elektroniczne systemy głosowania coraz bardziej zyskują na popularności.
Wraz z postępem technologii oraz coraz większego znaczenia internetu w życiu 
społecznym, pojawiają się pomysły (a nawet implementacje) przeniesienia procesu głosowania do sieci. Pomysły i implementacje internetowych systemów wyborczych dotyczą wyborów na szczeblu państwowym (przykładowo wybory prezydenckie), jak i wyborów organizowanych na potrzeby prywatne. System e-votingu zazwyczaj sprowadza się do serwisu internetowego, który pozwala na oddanie głosu poprzez odpowiednią stronę internetową. 

Przeniesienie głosowania do aplikacji internetowej, pozwala zminimalizować wpływ ewentualnego błędu ludzkiego podczas przeprowadzania wyborów. Jednakże takie rozwiązanie generuje nowe problemy, z którymi muszą się zmierzyć projektanci tych systemów. Największy problem stanowi zabezpieczenie aplikacji, przed zewnętrznymi próbami fałszowania wyników głosowania. Rozwój technologii prowadzi również do powstawania nowych odmian “złośliwego oprogramowania”, co prowadzi do ciągłego aktualizowania zabezpieczeń. Aplikacja odpowiadająca na potrzeby wyborów, na wysokim szczeblu powinna być tworzona “na potrzeby danych czasów”, bądź łatwa w płynnej aktualizacji.

Zabezpieczenie przed fałszowaniem głosów to nie jest jedyny problem, z jakim trzeba się zmierzyć podczas próby przeniesienia procesu głosowania do internetu. Kolejnym problemem jest sama logika systemu głosowania, niektóre wybory wymagają, aby informacje o wyborcach oraz ich głosach, były tajne. System powinien również umożliwiać weryfikację użytkownika, na podstawie dostarczonych przez organizatora wyborów danych logowania. Coraz głębsza analiza problemu generuje coraz więcej potrzeb z zakresu bezpieczeństwa systemu. Dodatkowo wyszczególniając elementy, które powinny podlegać szczególnej protekcji należy zadbać, aby architektura systemu pozwalała na łatwą aktualizację zabezpieczeń.

Wykorzystując komputery do obsługi głosowania, oczekuje się szybkiego i poprawnego uzyskania rezultatu głosowania. System taki powinien być zoptymalizowany, a czas uzyskania rezultatów powinien być deterministyczny. Warto rozważyć także moduł generujący statystyki wyborcze.

System wyborczy to nie tylko serwer, który zbiera, przechowuje i liczy głosy. Kliencka część aplikacji (widoczna dla wyborcy) powinna być responsywna, przejrzysta oraz przyjazna dla osób z pewnymi niepełnosprawnościami (głównie należy uwzględnić wady wzroku).

Dużą zaletą e-votingu jest przeniesienie procedur, które musi wykonać organizator do interaktywnej aplikacji przeglądarkowej. Aplikacja webowa powinna zapewniać możliwość zarządzania głosowaniem oraz kreator głosowania, ten element również powinien podlegać zabezpieczeniu danych i autentykacji użytkownika. Aplikacja wyposażona w takie funkcjonalności powinna spełniać wymagania elektronicznego systemu głosowania.

\subsection{Wstępne wymagania niefunkcjonalne}
Z powyższej analizy można wypunktować następujące wymagania:
\begin{itemize}

\item Dbałość o zabezpieczenie głosu wyborcy przed fałszerstwem - rezultat głosu nie może zostać zmieniony, po jego zatwierdzeniu.

\item Zabezpieczenie danych wyborcy, poprzez utajnienie jego tożsamości.

\item Sprawne i bezbłędne liczenie głosów - czas liczenia głosów powinien być deterministyczny.

\item Dbałość o prezencję aplikacji, strona kliencka powinna być przejrzysta i łatwa w obsłudze.

\item System musi być konfigurowalny, z uwzględnieniem bezpieczeństwa konfiguracji.
\end{itemize}


\subsection{Przegląd istniejących rozwiązań e-votingu}
Dokonana analiza jest wstępną analizą, która nie jest wystarczająco szczegółowa. Idealna aplikacja, jest dopracowana w najmniejszych detalach. W celu doprecyzowania wymagań można dokonać przeglądu i analizy istniejących rozwiązań. Analizując istniejące aplikacje można wyciągnąć wiele przydatnych wniosków.
\subsubsection{System estoński}
Przykładem regularnego wykorzystania e-votingu jest Estonia. Estonia jest krajem, który pierwszy udostępnił możliwość głosowania elektronicznego w wyborach lokalnych, na szczeblu krajowym. W pierwszych wyborach wzięło udział około 1\% wyborców. Od roku 2005 w Estonii postępował rozwój systemów e-votingu. Wraz z następnymi wyborami zwiększała się liczba uczestników. W roku 2019 liczba wyborców, którzy skorzystali z e-votingu wyniosła 43,8\% wyborców.

Na system e-votingu  w Estonii składa się kilka mniejszych rozwiązań, które można wylistować w następujący sposób:
\begin{itemize}

    \item Identyfikacja za pomocą karty e-obywatela lub profilu e-obywatela w telefonie.

    \item Architektura rozwiązania pozwala na wykonanie wielu głosów, a głosem wiążącym jest zawsze głos finalny.

    \item Serwery wykorzystywane podczas głosowania są pod szczególną ochroną i nie można uzyskać do nich dostępu z bezpośrednio z internetu (zabezpieczenie firewallem).

    \item Wykorzystywanie prywatnych kluczy i narzędzi kryptograficznych w celu zabezpieczenia dostępu do danych. Wykorzystywanie standardu SSL.
    
\end{itemize}


\end{document}
